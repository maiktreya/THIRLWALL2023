\documentclass{article}
\usepackage[rightcaption]{sidecap}

\usepackage{unicode}
\usepackage{imakeidx}
\usepackage{graphicx}
\usepackage{rotating}

\usepackage[utf8x]{inputenc}
\usepackage[T1]{fontenc}
\usepackage{indentfirst}
\usepackage{enumerate}
\usepackage[sectionbib,round]{natbib}
\usepackage[nottoc]{tocbibind}
\usepackage[titles]{tocloft}
\usepackage[flushleft]{threeparttable}
\renewcommand{\baselinestretch}{1.5}

\setlength{\parindent}{2em}
\setlength{\parskip}{1em}
\bibliographystyle{plainnat}
\graphicspath{ {IMAGES/} }

\title{BALANCE OF PAYMENTS CONSTRAINT AND GROWTH: EVIDENCE FROM THE EURO AREA(1992-2019)}
\author{Miguel García Duch}
\date{April 2021}

\begin{document}
\maketitle

\begin{abstract}
  This article estimates a balance of payments constrained growth model for the original 12 Eurozone -EZ- members but Luxembourg during a period spanning almost three decades (1992-2019). The estimation is performed taking advantage of the properties of cointegration following the ARDL bounds test approach. Our estimation procedure also took into consideration the important effect of the crisis exploded in 2008  allowing for the presence of structural breaks if necessary. Evidence strongly suggests that the so-called Thirlwall law, in spite of its simplicity, constitutes an accurate predictor of actual growth rates both for individual countries and the union as a whole.Therefore, suggesting the suitability of BPC models to understand the importance of external imbalances as mechanisms leading to the recent European crisis.
  \vspace{5mm} %5mm vertical space
  \hrule
  \vspace{5mm} %5mm vertical space

  Keywords: Eurozone; Balance-of-Payments; Thirlwall's Law, ARDL
  \vspace{3mm} %5mm vertical space
  \\JEL classification: E12; O41.
\end{abstract}
\clearpage

\tableofcontents
\clearpage


\section{INTRODUCTION}



\section{THE MULTI-SECTOR BPCGM}

For simplicity we are going to derive the core of BPCGM in both its standard and multisector from a common set of equations. Assume an open economy constituted by n distinct sectors where foreign trade is the only truly autonomous component of aggregate demand. As in the original version of the model developed in \cite{thirlwall1979balance}, assuming standard Cobb-Douglas demand functions, the resulting sector-demand equations of imports and exports for each sector will be defined in the following manner:
\begin{equation}
  X_{it}=x_{io}.\left(\frac{E_t.P_{fit}}{P_{it}}\right)^{\gamma _{i1}}.Z_{it}^{\varphi_i}
\end{equation}
\begin{equation}
  M_{it}=m_{io}.\left(\frac{P_{it}}{E_t.P_{fit}}\right)^{\gamma _{i2}}.Y_{it}^{\Psi_i}
\end{equation}
Where  $X_{it}$ and $M_{it}$ are volume of exports and imports respectively, evaluated in real terms, for a certain moment $t$ and sector $i$.  $({E_t.P_{fit}}/{P_{it}})$ stands for the evolution of sector -domestic- relative prices - $P_{fit}$ foreign prices, $P_{it}$ national prices, and exchange rate $E_t$. Thus $\gamma _{i1}$ and $\gamma _{i2}$ are price-elasticities of demand for exports and imports. Respecting income-elasticities of demand $Z_{it}$ represents foreign income and $\varphi_i$ is its associated elasticity. Finally, $Y_{it}$ , national income and $\Psi_i$ for its elasticity. On the other hand consider the usual current account equilibrium identity defined, in nominal terms and for a certain period $t$ as:
\begin{equation}
  P_t.X_t = P_{ft}.E_t.M_t
\end{equation}

Equation (3) implicitly assumes TL main lemma: in the long run, growth is BPC and the the current account has to be, in average, balanced. Is straightforward to connect sector-demand equations defined in (1) and (2) to the  aggregate current account identity exposed in (3) through cumulative sector-sums. Following this strategy, aggregate level of imports in nominal terms will be unfolded as $P_{ft}M_{t} = \sum_{i=1}^{n} ({P_{fit}}M_{it})$ and just rearranging we will get:
\begin{equation}
 M_t = \sum_{i=1}^{n} (\frac{P_{fit}}{P_{ft}}M_{it}) = \sum_{i=1}^{n} M_{it}
\end{equation}
In the same manner, aggregate exports are represented by $P_{t}X_{t} = \sum_{i=1}^{n} ({P_{it}}X_{it})$ and after the same transformation applied to imports we have:
\begin{equation}
 X_t = \sum_{i=1}^{n} (\frac{P_{it}}{P_{t}}X_{it}) = \sum_{i=1}^{n}X_{it}
\end{equation}


This set of 5 equations is all we need to develop both the original and multi-sector version of the BPCGM. First, as a prerequisite to easily evaluate the dynamics and solution of the model we have to transform it from levels to growth rates.  taking logarithms and first time differences of equations (1),(2) and (3) we get:

\begin{equation}
  m_i= \gamma_{i2}(p_i-e-p_{fi}) +\Psi_i y_i
\end{equation}
\begin{equation}
  x_i= \gamma _{i1}\left(p_{fi}+e-p_i\right)+\varphi_i z_i
\end{equation}
\begin{equation}
p + x  =m+p_{\mathit{f}}+e
\end{equation}

Where lower case variables stand for growth rates of corresponding variables in levels. Now, assume that for simplicity we only have one economic sector. Assuming $(n=i=1)$, we can solve for $y$ the system constituted by (6), (7), and (8). After some algebraic manipulation we get:

\begin{equation}
  y_{BPC}= \frac{ (\gamma_1 +\gamma_2 -1) (p_{f}+e-p)+  \varphi z }{\Psi}
\end{equation}

Assuming, as evidence suggests, that in the long run price effects are neglible as a result of either PPA holding $(p_{fi}+e-p_i)=0$ or price competitiveness unimportance through the Marshall-Lerner condition being just met $(\gamma_1 +\gamma_2 -1)=0$, equation (9) can be reduced to:
\begin{equation}
  y_{TL}= \bar{z} \frac{  \varphi  } {\Psi}
\end{equation}

This equation is the original form, or "strong" version of TL. Finally, if we substitute the upper part of (10) directly by exports growth, the only estimated parameter would be $\Psi$ leading to the final form of Thirlwall Law in its so-called "weak" version.
\begin{equation}
  y_{WTL}=\frac{x}{\Psi}
\end{equation}

Let's turn over the multisectoral version of the model originally developed by \cite{araujo2007}. Recall equations (6) and (7) for sectoral demands in log-differeces and assume now $(n>1)$ and that sectoral relative prices  growth follow in average the PPA postulate as exposed above so that prices effects become irrelevant. As a result we can easily define aggregate equations derived from combining (4) with (6) and (5) with (7) respectively. Aggregate growth of exports and imports in the MS-BPCGM are then defined as:

\begin{equation}
  x= \bar{z} \sum_{i=1}^{n}(\mu_{i}\varphi_{i})
\end{equation}
\begin{equation}
  m= \bar{y} \sum_{i=1}^{n}(\lambda_{i}\psi_{i})
\end{equation}

Where $\mu_{i}$ and $\lambda_{i}$ represent individual sectoral shares, that is:
\begin{equation}
 \sum_{i=1}^{n}\mu_{i}=1  \quad \forall  \quad	\mu_{i} = X_{it} / X_{t}
\end{equation}
\begin{equation}
\sum_{i=1}^{n}\lambda_{i}=1 \quad \forall \quad	 \lambda_{i}= M_{it} / M_{t}
\end{equation}

Finally, Thirlwall law version in a multi-sector context becomes, substituting equations (12) and (13) in the original one sector TL defined in (10):

\begin{equation}
  y_{MTL} = {\displaystyle \bar{z}} \frac{\displaystyle \sum_{i=1}^{n}(\mu_{i}\varphi_{i})  }{\displaystyle \sum_{i=1}^{n}(\lambda_{i}\psi_{i}) }
\end{equation}

Equation (16) is assumed to be a more robust version of (10) which takes into consideration additional factors as sector structure or technological specialization. Therefore, it generally offers both a more complete picture and almost always stronger forecasts of income growth than traditional TL.

\section{THE EUROZONE AND ITS IMBALANCES}
The Eurozone construction process and its culmination with the adoption of the Euro constituted an important landmark in regional integration processes. This process have to key policy implications for the countries involved in the process: a) the loss of control over the exchange rate and as a result the possibility of competitive devaluations and b) the adoption of a centralised reference interest rate. Both dimmensions where controlled since by the European Central Bank. The implications of such changes for a group of country with higly heterogeneous productive structures , specially between, generally speaking, two different group of countries joining the original Euro country: mediterranean and central economies. The consequences of these policies for the deepness and length of the Euro-crisis of 2011-2012 following the great recession started in 2007-2008 have been analysed in depth elsewhere (\cite{stockhammer}).

\section{METHODOLOGY AND DATA SOURCES}

\section{EVIDENCE}

\section{CONCLUDING REMARKS}

\bibliography{references.bib} % generate bibliography

\end{document}
